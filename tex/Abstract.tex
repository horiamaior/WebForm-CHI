\documentclass[../main/Feedback.tex]{subfiles}
\begin{document}
\begin{abstract}
Amongst the many tasks in our lives, we encounter web forms on a regular basis, whether they are mundane like registering for a website, or complex and important like tax returns. Amongst the many aspects of Usability, one concern for user interfaces is to reduce mental workload and error rates. Whilst most assessment of mental workload is subjective and retrospective reporting by users, we examine the potential of fNIRS as a tool for objectively and concurrently measuring the mental workload of particiapnts. We use this technology to examine the design of three different form designs for a car insurance claim, and show that a form divided into subforms increases mental workload, contrary to our expectations. We conclude that fNIRS is highly suitable for objetively exmaining mental workload during usability testing, and will therefore be able to provide more detailed insight than summative retrospective assessments. Further, for the fNIRS community, we show that the technology can easily move beyond typical psychology tasks, and be used for more natural study tasks.
%	The study aimed to improve the web form filling process and produce implications for design. In order, to provide generalisable results we tested 3 different layouts of a insurance claim web form. We tested dividing a standard web form into 3 separate web pages, and putting the description box at the beginning instead of the end of the form. Because it is suggested by usability design heuristics that workload should be reduced as much as possible we decided to obtain both objective and subjective inferences about workload. We used functional near infrared spectroscopy(fNIRS) as objective measure, and NASA-TLX as subjective. The result showed that the divided page web form yielded the highest workload, while the standard single paged web form the least. We have discussed the results and provided implications for design. In addition, fNIRS practicality for evaluation of web form interfaces was assessed as limited.
\end{abstract}
\end{document} 