\documentclass[../main/Feedback.tex]{subfiles}
\begin{document}
\begin{abstract}
	In this dissertation we evaluated a web interface using functional near infrared spectroscopy(fNIRS), and verified the practicality of this brain imaging modality. More specifically, we tested the web page layout of an online insurance claim process. Three variations of the web forms were created: one standard insurance claim form, one which has alternated order of form fields, and one which divides the forms into three pages. We hypothesised that there should be significant difference between them in the objective and subjective measures of mental workload. Also, we tried to elicit emotional state from the fNIRS data and hypothesised that a correlation will be observed between the objective and subjective measures of emotional valence. We found that the control condition elicited the lowest mental workload and positive valence according to the majority of the corresponding measures. In contrast, the divided page approach evoked highest mental workload and positive valence. We contribute to web form design and usability research by proposing implications for design. In addition, we assessed the practicality of fNIRS as limited particularly for low engaging web interfaces due to the higher amount of time, and resources required to run and analyse the data.
\end{abstract}
\end{document} 