\documentclass[../main/Replicate.tex]{subfiles}
\begin{document}
\section{Discussion}
	\subsection{Comparison between the three web forms}
%	\subsubsection{Control condition - Index1}
%	The purpose of this study was to improve usability of web form filling of insurance claims and find which of three web forms layouts was the most usable. When analysing the results it can be noted that index1 was the least mentally demanding web form, according to the mental demand scale of NASA-TLX, and objective measure of mean Hbo activation. However, the calculated total tlx indicated higher perceived workload than index2, and lower than index3. Index1 can be said to support the consistency heuristic\cite{nielsen1990heuristic} because it was the control condition and represented a standard insurance claim form. The comparatively low workload measured for index1 can be attributed to the fact that the task was familiar, and therefore participants had to spending less working memory resources on planning because the task sequence was already mapped in their brain. One participant supported this argument by saying: \textit{``I consider it more logical, like logical approach..to say who you are, what happened to your car and then what happened exactly, describe it from your own perspective.''}. Also, 4 users reported that the sequence of small question first, leading to the description at the end, helped them to recall the situation better: \textit{``I preferred the first(index1) and third(index3) examples, as they asked more general questions at the beginning. This made me think about what aspects to include in the event summary at the end, such as number of cars involved''}. Which means that the sequence of web forms or their order of presentation is important aspect to consider when designing long web forms.
%	
%	Index1 induced the least perceived emotional valence from the SAM questionnaire, compared to the rest of the conditions, with marginal significance of $p=0.073$ when compared to index3. However, the objective data from the fNIRS differences demonstrated higher left hemisphere activation than index2, and slightly less than index3. Similarly, it took more time to complete the web form compared to index2, and slightly less time than index3.	
%	
%	\subsubsection{Description at beginning - Index2}
%	Generally, index2 elicited the least workload inferred from the mean Hbo concentration, total tlx, perceived effort, physical demand, and frustration measures(Table \ref{nasa-tlx}). Also, users performance was higher in index2, as the task completion time was the lowest, although not statistically significant(see Fig. \ref{fig:mean-task-completion-time}). One interpretation is that index2 supported participants working memory better, specifically for this study design, because index2 presents the description field first, thus lowering the time to hold the visual information in the episodic memory. Consequently, users memory was relatively fresher when filling the description field in index2, which demanded the most attentional capacity, compared to the rest of the conditions. 
%	
%	Furthermore, index2 elicited approximately neutral emotional valence rating(3.40) which was higher than index1(3.10) and lower than index3(3.70), see Figure \ref{fig:sam-valence-index123}. In contrast, the objective data from fNIRS differences reveal more right hemisphere activation which can be interpreted as more negatively valenced affect or avoidance motivation. The last finding can be supported with the fact that only 3 out of 20 participants expressed preference for index2. A possible explanation is that index2 did not support the consistency heuristic\cite{nielsen1990heuristic,shneiderman1992designing} because the description field is normally situated at the end of insurance claim forms. This statement can be supported by the following participant comments: \textit{``The only one I didn't like is the first one(index2) particularly. It was just describing the event before filling out the details it is a bit awkward.''}, and another participant stated:\textit{``Index1 and Index3 were Ok, but I didn't like index2 because I was not used to it.''}.
%	
%	\subsubsection{Divided form - Index3}
%	We found out that index3 to elicits the highest mental workload according to the fNIRS Hbo data, SAM arousal scale(significant difference with index1), total tlx, mental demand scale from NASA-TLX. Also, it takes the most time to complete the task compared to the rest of the conditions. However, index3 was evoking the highest positive valence(SAM), fNIRS differences were indicating more left hemisphere activation, and users preferred it the most when asked after the finish of the experiment. The results were contradictory according to the Norman\cite{norman2002emotion} positive affect should enhance operator creativity, thus increasing the performance.
%	
%	Also, it can be noted that perceived temporal demand for index3 was lower than index1 and index2(no statistical significance). This can be attributed to the fact that index3 had lower number of forms(visual cues) on each page. It can be inferred that the more web forms are present on an interface, the more the user perceives temporal demand because it appraises the situation as one that needs more work to be performed. This claim can be supported with the feedback from the participants: \textit{``so the second one(index3) I felt having to click the links broke it down a little bit, like you didn't have to think about everything in one go...''} or other user expressed \textit{``you give all the details and then you take your time to write about the incident so you don't feel very rushed or something''}. 
	
	\subsubsection{Divided vs single page approach}
	The purpose of this study was to improve usability of web form filling of insurance claims and find which of three web forms layouts elicited the least workload. A division can be made between the web forms which contain all forms on one page(index1, index2), and web forms which are consolidated or divided on several pages(index3). We define these as the \textit{single page approach}, and the \textit{divided page approach}.
	
	The single page approach elicited significantly less mental demand (objective and subjective) than the divided page approach. One explanation can be that with single page approach participants can go back and verify what information they have already entered: \textit{``I remember in the second one(index1) I'm not sure whether the option provided left or right, so I rechecked''}. This way working memory resources are saved because participants has the ability to quickly recheck what they have already entered, relying on recognition, rather than recall\cite{nielsen1990heuristic}. Another advantage of the single page approach is that participants can choose which form to start first: \textit{``the good thing about the number two(index1) is everything is on the same page I can choose whatever I like''}. Generally, some of the participants preferred to fill in the description field first, and then the rest of the forms, so researchers and practitioners have to give users the power to choose from where they can start. 
	
%	However, in our case the participants watched a recorded accident that they had to recall after approximately 2 minutes. In reality, the delay between the accident and the claim form filling will be much higher, thus the arrangement of the web forms depends on the time between the accident and the claim form filling.
	
	The results were contradictory to our expectations because we assumed that the divided page approach should reduce visual search and clutter, thus demand less attentional resources. Also, the more informational cues(web form fields) are present on an interface, the more time the user should spent on searching for information, thus the performance should drop. This claim can be supported with the feedback from the participants: \textit{``so the second one(index3) I felt having to click the links broke it down a little bit, like you didn't have to think about everything in one go...''}. However, the workload was the highest for the divided page approach. This can be attributed to the fact that some of the users did not liked navigating 3 pages to complete the process: \textit{And I like hate waiting for the next
	page, navigating to the next page so first one(index3)} due to \textit{the tasks were small, questions were small, so why do you need to navigate so many pages, two pages, three pages}. Which suggests that designers should take into consideration not just the number of fields but also their demand for working memory resources. This suggest that, if there is low number of form fields but they demand a lot of time and resources to complete, they could be divided into separate pages. In contrast, if we have high number of form fields which are easy to complete there could be no need for dividing them into separate pages.

	In our opinion, index1 is more suitable for implementation in our particular interface because it elicited less objective and subjective mental workload, and 8 out of 15 users expressed their preference towards it, and it took the least time for participants to complete the task. Single page approach was more suitable for this particular task, however, researchers should not infer that it is generally better than the divided page approach. As Norman stated [] every interface is influenced by different aspects like, context, skill level, culture, individual preferences, therefore every interface should be evaluated individually. 

%	\textit{``because you don't have to think about the other forms''}. 
	\subsection{Implications for Design}
	Based on the obtained results we propose the following recommendations for design of long web forms:
	\begin{itemize}
		\item Include visual representations of information where possible, rather than display it as text only:\\
		Some participants did not knew the names of the different body parts of a car and a few even used search engine to see what the different words like, bonnet and bumper meant due to language deficiency. To address this problem, we propose that a visualisation of car divided on its main body parts should be depicted on the form in order to support recognition. Thus, designers should offer visual representations of information which is difficult or it takes too much time to comprehend.
		
		\item Sequence of web forms is important aspect to consider, so testing the order of presentation of forms may yield important information:\\
		Some of the participants wanted to fill different form fields first, and scrolled down to find them. Consequently, researchers should consider the sequence of web forms: which questions to ask first, and which last in their web forms.
		
		\item Divide a long web forms onto separate pages if the information entered in each field is not relative to the other fields, or when the users do not need to check what they have entered previously:\\
		During the web form filling process it can be noted that participants occasionally inspect what they have written before, in order to verify that information or check what they had written before. 
	\end{itemize}
	
	\subsection{fNIRS practicality for HCI evaluations}	
	In conclusion, fNIRS provides useful data, however we can infer about design issues from user trials, and receive rich data for relatively less amount of money compared to the more expensive and time consuming fNIRS evaluation study. From the brain scanner we receive single continuous measure which can be interpreted in various ways, but still gives only limited data about the hemodynamic activity, which itself is not well understood in cognitive neuroscience. Consequently, fNIRS has limited practicality for evaluating neutral web interfaces, like web forms because it requires more time and financial resources, and gives relatively vague data, compared to the established methods, like user trials which are cheap to conduct, less time consuming and provide rich data. However, as noted above fNIRS appears to be particularly useful in game evaluation studies where the continuous measure will provide valuable information about the level of engagement\cite{harrivel2013monitoring} and the flow states\cite{yoshida2014brain} of players.	
	
%	\subsection{Limitations of the study and future work}
%	This study attempts to simulate real conditions, by simulating automotive accident and then presenting a insurance claim form to fill. Participants wait approximately 2 minutes after they watch a video to start filling the web form. In real situations the delay between the accident and the form filling is usually higher. Therefore, it lacks ecological validity. Watching videos of automotive accidents may also have biased the subjective ratings of participants when they had to rate the web forms. 
%	
%	It should be noted that we had difficulties with the fNIRS device, as some of the channels did not worked reliably. Therefore, we had to exclude a large amount of participant data that was defective or incomplete. As a result, we obtained less statistical power from the results. However, this was a problem encountered because of malfunctioning of the particular device, and not problem encompassing all fNIRS based systems. In addition, the fNIRS device which we used was slightly invasive and wearing it caused pain and discomfort, after approximately 30-40 minutes. As a result, we had to design our experimental protocol so that the duration does not exceed 40 minutes.
%	\subsection{Future work}
%	To produce more valid results, experiments without the inclusion of videos should be conducted. For example testing the insurance quote process, instead of claims. Also, the divided page approach should be further examined because it produced encouraging and positive results, except that it required the most attentional resources. However, as it is not clear which improves the user experience more, the workload experienced or the positive or negative appraisal of the interface, we should further test and diagnose the single versus divided page approaches.
	\subsection{Conclusion}
	In summary, we used functional near infrared spectroscopy in a user study of the web form filling process of insurance claim. We aimed to obtain objective and subjective measurements of mental workload and emotional valence. We tested the web page layout by producing 3 variations. After conducting the experiment the standard single page web form which was the control condition, showed the least mental workload, according to the objective and subjective measures that we used to infer it. On the other hand, the web form which was divided on three separate pages, yielded highest workload, but elicited most positively valenced ratings. However, we could not determine which of the web forms is more suitable for implementation, as we are uncertain which is more important for long forms: to elicit positive affect, or to be less mentally demanding. However, the answer is varies for the different interfaces and implementations. Consequently, experiments should be conducted to examine the difference between single and divided page approaches for long web forms in more details. Also, we assessed the practicality of fNIRS for similar web interface studies to be limited.  Finally, we outlined implications for design for long web forms, and discussed the disadvantage of the study.
\end{document}